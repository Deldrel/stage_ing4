%! Author = pierre
%! Date = 28/06/2024

\documentclass[10pt]{article}

\usepackage[utf8]{inputenc}
\usepackage{amsmath}
\usepackage{amsfonts}
\usepackage{amssymb}
\usepackage{graphicx}
\usepackage{hyperref}
\usepackage{natbib}
\usepackage{geometry}
\usepackage{tikz}
\usepackage{pgfplots}
\usepackage{subcaption}
\usepackage{comment}
\usepackage{tabularx}
\usepackage{helvet}
\usepackage{float}
\usepackage{booktabs}
\usepackage{array}
\renewcommand{\familydefault}{\sfdefault}
\pgfplotsset{compat=1.17}

\geometry{a4paper, margin=1in}

\hypersetup{
    colorlinks=true,
    linkcolor=blue,
    urlcolor=black,
    citecolor=blue,
}

\title{Premier rendu: Abstract}
\author{Pierre LAPOLLA}
\date{\today}

\begin{document}

    \maketitle
    \begin{abstract}
        Itinerary recommendation systems are used on a daily basis by a wide range of people, to go from a starting place to their destination.
        However, most existing systems do not consider accessibility constraints specific to wheelchair users.
        There is also not much data available on wheelchair users itinerary.
        In this context, this paper proposes an adaptation and evaluation of an existing itinerary recommendation system: Diffusion Convolutional Recurrent Neural Network (DCRNN).
        This adaptation is capable to handle multiple locomotion modes, specifically cars and wheelchairs, and computes the time required to travel from point A to point B.
        Besides the model adaptation, this study proposes a method to derivate wheelchair traffic data from a dataset of car traffic data.
        Finally, the evaluation of the model is performed through the comparation to a baseline using the Mean Absolute Error (MAE), Mean Absolute Percentage Error (MAPE) and Mean Squared Error (MSE) error metrics.
        The source code is available at \url{https://github.com/Deldrel/stage_ing4}.
    \end{abstract}

\end{document}