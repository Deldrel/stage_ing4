The need for itinerary recommendation systems has been significantly met by large platforms like Google Maps, Waze, and Apple Maps.
They offer a wide range of functionalities, including itineraries for mobility-impaired users.
However, accurately anticipating travel times for various kind of locomotion modes is still a challenge.
Additionally, these companies collect private data, raising concerns about user privacy.
To address these issues, there is a growing need for an open-source model that can provide comprehensive itinerary
    recommendations for all modes of transportation, including car, bike, wheelchair and walking.
\vspace{1em}

Multi-modal traffic prediction faces multiple challenges.
The first one is about data sources, as data collection on multiple modes of transportation within the same area has never been done before.
The artificial intelligence model is the second challenge.
Managing various locomotion modes and capturing the temporal and spatial dependencies of traffic is a challenging task.
Finally, the third challenge is about scalability.
Scalability is needed for accurate real-time traffic prediction.
\vspace{1em}

To demonstrate the feasibility of multi-modal traffic prediction, wa have included wheelchair users as a specific case study.
As they often faces unique challenges in mobility, such as inaccessible roads or sidewalks.
By adding this mode of transportation, we highlight the adaptability of our model and make a meaningful social contribution.
\vspace{1em}

This paper proposes an adaptation of the Diffusion Convolutional Recurrent Neural Network (DCRNN) model~\cite{DCRNN} to handle the complexity of multi-modal traffic prediction.
Our approach includes:
\begin{itemize}
    \item \textbf{Model Adaptation}:
    Enhancing the DCRNN model to take into account multiple locomotion modes, including cars and wheelchairs.
    \item \textbf{Data Derivation}:
    Developing a method to derive wheelchair traffic data from car traffic data in the METR-LA dataset.
\end{itemize}
