The need for itinerary recommendation systems has been significantly met by large platforms like Google Maps, Waze, and Apple Maps.
They offer a wide range of functionalities, including itineraries for mobility-impaired users.
However, accurately anticipating travel times for various kind of locomotion modes is still a challenge.
Additionally, these companies collect private data, raising concerns about user privacy.
To address these issues, there is a growing need for an open-source model that can provide comprehensive itinerary
    recommendations for all modes of transportation, including car, bike, wheelchair and walking.

Multi-modal traffic prediction faces multiple challenges:
\begin{itemize}
    \item \textbf{Data Sources}:
    Collecting traffic data on multiple modes of transportation in the same area has never been done before.
    \item \textbf{Modeling}:
    The model needs to capture both spatial and temporal dependencies of traffic while also being able to handle multiple modes of transportation.
    \item \textbf{Scalability}:
    Accurate real-time traffic prediction over a vast area requires a scalable model.
\end{itemize}

To demonstrate the feasibility of multi-modal traffic prediction, wa have included wheelchair users as a specific case study.
As they often faces unique challenges in mobility, such as inaccessible roads or sidewalks.
By adding this mode of transportation, we highlight the adaptability of our model and make a meaningful social contribution.

This paper proposes an adaptation of the Diffusion Convolutional Recurrent Neural Network (DCRNN) model~\cite{DCRNN} to handle the complexity of multi-modal traffic prediction.
Our approach includes:
\begin{itemize}
    \item \textbf{Model Adaptation}:
    Enhancing the DCRNN model to take into account multiple locomotion modes, including cars and wheelchairs.
    \item \textbf{Data Derivation}:
    Developing a method to derive wheelchair traffic data from car traffic data in the METR-LA dataset.
\end{itemize}
