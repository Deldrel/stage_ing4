Our article introduced a new method for multi-modal traffic prediction that takes into account both car and wheelchair traffic by modifying the DCRNN model.
We used the METR-LA dataset to derive wheelchair traffic data from car traffic data, solving the lack of wheelchair traffic data.
With the added attention mechanism and modified convolution operation, our model performed as intended on car and wheelchair data independently.
However the model's performance on multi-modal traffic prediction was not as good as expected.
This behavior could come from our methodology, either from the model architecture or the data derivation.
This study establishes new ideas for research around inclusivity in traffic prediction.
In order to contribute to the research community, we offer recommendations for further accessibility data collection and model improvements.