The results of our adapted model are promising but also reveal areas that require further research.
\begin{itemize}
    \item \textbf{Car Traffic Prediction}:
    Our model achieve a better performance than the DCRNN baseline in terms of MAE\@.
    This confirms that our modifications are not penalizing the model's ability to handle car traffic data.
    \item \textbf{Wheelchair Traffic Prediction}:
    There is no baseline for wheelchair traffic prediction, as this is a novel contribution of our study.
    The results are promising and suggest that the model can effectively handle wheelchair data.
    The MAE is very low compared to the car traffic prediction, this indicates better accuracy but needs to be contextualized.
    As our speed feature is not normalized and wheelchair speeds are significantly lower than car speeds, the MAE is expected to be lower.
    When normalized, both MAE are comparable, even if wheelchair MAE remains slightly lower.
    \item \textbf{Multi-Modal Traffic Prediction}:
    The model's performance on multi-modal traffic prediction, combining cars and wheelchair data, is not as good as expected.
    This confirms the complexity of handling multiple locomotion modes with a single model.
    The increase of the MAE could come from the data and/or the model architecture.
\end{itemize}

Our study introduces implications and limitations that come from our methodology.
\begin{itemize}
    \item \textbf{Data Derivation}:
    The approach of deriving wheelchair traffic data from car traffic data adds a new methodology.
    This derivation may include many biases.
    Even if this method is innovative, it highlights the need for dedicated data collection targeting other locomotion modes than cars.
    \item \textbf{Scalability and Real-World Application}:
    The model's performance over a large dataset (METR-LA) suggests potential scalability.
    However, the variation in results across different modes of transportation indicates that real-world application will require
        more complex data handling and possibly the integration of real-time data.
    \item \textbf{Model Generalization}:
    While the model performs well on the METR-LA dataset, its generalization to other areas remains untested.
    Further efforts are needed to evaluate the model across varied datasets.
\end{itemize}

Looking ahead, we propose several directions for future work.
\begin{itemize}
    \item \textbf{Data Collection}:
    Conducting new data collection efforts could benefit for our model, but also for the research community.
    \item \textbf{Model Improvement}:
    Further improvements to the model architecture could enhance its performance on multi-modal traffic prediction.
    Adding other techniques or dividing the prediction task between multiple models are possible directions of research.
    \item \textbf{Adding Accessibility Features}:
    Adding more accessibility features to the data, such as public transport accessibility, could emphasize the social contribution of our study.
    \item \textbf{Model Comparison}:
    We could improve our understanding of the model's performance by comparing it to other models, and measuring its performance on different metrics and horizons.
\end{itemize}
