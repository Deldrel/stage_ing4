%! Author = pierre
%! Date = 16/04/2024

\documentclass[conference]{IEEEtran}
\IEEEoverridecommandlockouts

\usepackage{cite}
\usepackage{amsmath,amssymb,amsfonts}
\usepackage{graphicx}
\usepackage{textcomp}
\usepackage{xcolor}
\usepackage{hyperref}
\usepackage{enumitem}
\usepackage{pdfpages}

\hypersetup{
    colorlinks=true,
    linkcolor=blue,
    urlcolor=black,
    citecolor=blue,
}

\begin{document}

    % \includepdf{resources/cover.pdf}

    % \onecolumn

    \title{AI and Itinerary Optimisation}
    \author{
        \IEEEauthorblockN{Pierre LAPOLLA}
        \IEEEauthorblockA{
            \textit{LyRIDS}\\
            Paris, France \\
            pierre.lapolla@edu.ece.Fr
        }
        \and
        \IEEEauthorblockN{Guilherme MEDEIROS MACHADO}
        \IEEEauthorblockA{
            \textit{LyRIDS}\\
            Paris, France \\
            gmedeirosmachado@ece.fr
        }
        \and
        \IEEEauthorblockN{Frédéric RAVAUT}
        \IEEEauthorblockA{
            \textit{LyRIDS}\\
            Paris, France \\
            frederic.ravaut@ece.fr
        }
    }

    \maketitle

    \begin{abstract}
        Itinerary recommendation systems are used on a daily basis by a wide range of people, to go from a starting place to their destination.
        However, most existing systems do not consider accessibility constraints specific to wheelchair users.
        There is also not much data available on wheelchair users itinerary.
        In this context, this paper proposes an adaptation and evaluation of an existing itinerary recommendation system: Diffusion Convolutional Recurrent Neural Network (DCRNN).
        This adaptation is capable to handle multiple locomotion modes, specifically cars and wheelchairs, and computes the time required to travel from point A to point B.
        Besides the model adaptation, this study proposes a method to derivate wheelchair traffic data from a dataset of car traffic data.
        Finally, the evaluation of the model is performed through the comparison to a baseline using the Mean Absolute Error (MAE), Mean Absolute Percentage Error (MAPE) and Mean Squared Error (MSE) error metrics.
        The source code is available at \url{https://github.com/Deldrel/stage_ing4}.
    \end{abstract}
    % \tableofcontents


    \section{Introduction}\label{sec:introduction}

    \begin{itemize}
        \item Clearly state the problem and its significance
        \item Introduce the specific challenges faced by wheelchair users regarding itinerary planning.
        \item Briefly mention the proposed solution, DCRNN adaptation, and its potential impact.
    \end{itemize}


    \section{Related Work}\label{sec:related-work}

    \begin{itemize}
        \item Summarize existing research on itinerary recommendation systems.
        \item Discuss studies focused on accessibility and wheelchair user mobility.
        \item Highlight the gap in current research that your paper aims to fill.
    \end{itemize}


    \section{Methodology}\label{sec:methodology}

    \begin{itemize}
        \item Detail the adaptation process of the DCRNN model.
        \item Describe the method used to derive wheelchair traffic data from car traffic datasets.
        \item Explain the evaluation metrics used (MAE, MAPE, MSE).
    \end{itemize}


    \section{Results}\label{sec:results}

    \begin{itemize}
        \item Present the results of your model's performance compared to the baseline.
        \item Use tables and figures to illustrate key findings.
        \item Discuss the significance of your results in the context of the problem.
    \end{itemize}


    \section{Discussion}\label{sec:discussion}

    \begin{itemize}
        \item Interpret the results and their implications.
        \item Discuss any limitations or unexpected findings.
        \item Suggest areas for future research.
    \end{itemize}


    \section{Conclusion}\label{sec:conclusion}

    \begin{itemize}
        \item Summarize the main contributions and findings of your paper.
        \item Reiterate the significance of addressing accessibility in itinerary planning.
        \item Highlight the potential impact of your work and suggest future directions.
    \end{itemize}

    \section*{Acknowledgment}
    The authors would like to thank Thomas ROUSTAN for their valuable assistance and support during this project.

    \bibliographystyle{unsrt}
    \bibliography{report/references}

\end{document}